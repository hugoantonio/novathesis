%!TEX root = ../template.tex
%%%%%%%%%%%%%%%%%%%%%%%%%%%%%%%%%%%%%%%%%%%%%%%%%%%%%%%%%%%%%%%%%%%%
%% abstrac-pt.tex
%% NOVA thesis document file
%%
%% Abstract in Portuguese
%%%%%%%%%%%%%%%%%%%%%%%%%%%%%%%%%%%%%%%%%%%%%%%%%%%%%%%%%%%%%%%%%%%%

Culturas agrícolas são alvo de inúmeras pragas e mudanças climatéricas que podem causar danos económicos. É essencial recolher estes dados e prever potenciais ocorrências de  problemas nestas culturas.

Para solucionar um método de recolha de dados inadequado, cadernos de campo, esta dissertação visa desenvolver uma framework na forma de aplicação que permita registar dados e eventos na plantação, geovisualizar o estado das plantas e monitorizar as pragas inimigas da cultura plantada.

Desde estações meteorológicas, sensores de solo e observações biológicas de pragas, dados vão ser coletados do campo de cultivo para o servidor e devolvidos ao utilizador/agricultor na forma de uma visualização digital interativa: Mapas de risco. Todos os dados coletados devem ser geo-referenciados e serão obtidos com dispositivos móveis (smartphones) e/ou sensores \emph{in situ}, documentados com fotografias, vídeos e texto sempre que necessário. Esta dissertação descreve e apresenta uma \emph{framework} que emprega a metodologia da Agricultura de Previsão que é destinada a recolher informação e calcular métricas relevantes sobre os dados de diferentes fontes sensoriais de uma forma escalável.

A solução apresentada neste documento deve auxiliar agricultores na tomada de melhores decisões ao gerir as suas propriedades agrícolas, ajudar a documentar o comportamento de algumas espécies de insetos e reduzir a perda de produção dos parceiros do grupo operacional FitoAgro. 

% Palavras-chave do resumo em Português
\begin{keywords}
Pragas, mudanças climatéricas, recolha de dados, cadernos de campo, geovisualização, framework, estações meteorológicas, dispositivos móveis, sensores, observações biológicas, agricultura de precisão, FitoAgro \ldots
\end{keywords}
% to add an extra black line
