%!TEX root = ../template.tex
%%%%%%%%%%%%%%%%%%%%%%%%%%%%%%%%%%%%%%%%%%%%%%%%%%%%%%%%%%%%%%%%%%%%
%% abstrac-en.tex
%% NOVA thesis document file
%%
%% Abstract in English
%%%%%%%%%%%%%%%%%%%%%%%%%%%%%%%%%%%%%%%%%%%%%%%%%%%%%%%%%%%%%%%%%%%%
Crops are subject to numerous pests and weather variables that may cause significant economic damage. It is essential to aggregate this data and forecast the potential occurrence of the major problems in these crops. 

To fix an inadequate data collection process, the goal of this dissertation is the development of a framework in the form of platform for data/event registering and geovisualization of plant status and pest monitoring.

Ranging from weather stations, soil sensors and pest biological observations; data will be collected from the field to the server and given back to the user/farmer in the form of a rich digital visualization: Risk Maps. All the data collected shall be geo-referenced and will be obtained with mobile devices or sensors \emph{in-situ}, documented with photos, videos and text whenever needed.
This thesis describes and presents a fully working framework that employs the methodology from Precision Agriculture that is able to collect data and compute metrics from the several sources of information available, in a scalable way.

The solution presented will empower farmers to make better decisions when managing their properties, help documenting specific species behaviour and reduce yield loss for the partners in the FitoAgro operational group.

% Palavras-chave do resumo em Inglês
\begin{keywords}
Framework, Data Collection, Precision Agriculture, Geovisualization, mobile devices, sensors, Web Application, Pest Monitoring, Risk Maps, FitoAgro.
\end{keywords} 
